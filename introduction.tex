Melanoma is one of the rarest forms of skin cancer, yet it is also the most dangerous one, because of its ability to rapidly spread.
\newline
Melanoma can be treated with a high probability of survival if discovered early; however, this can be a challenge because of similarly looking skin conditions and long queues for dermatologists. As such, scientists have turned to machine learning models, to try and assist with assessing skin lesions \cite{SCF_Melanoma}.
\newline
We use the PAD-UFES data set \cite{pacheco2020padufes}. Previously this data set has been analyzed by extracting the features and combining it with patient information using neural networks \cite{PACHECO2020103545}. With this report, we seek to instead investigate whether melanoma can be categorized automatically based solely on the colour, asymmetry, and appearance of blue-white veils in the lesions.