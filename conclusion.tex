An important aspect to consider is that melanoma is typically classified based on more available data, such as the patient data combined with neural networks \cite{pacheco2020padufes} or based on the 7 features \cite{seven-pointchecklist}, among which blue-white veil is one of them. However, we are trying to correctly classify melanoma based on a limited number of features compared to the amount typically used, which shows in the results.\\
\newline
Another aspect to consider is the limitations of our coding abilities. Upon initial inspection of the images, we discovered irregular blotches in many melanoma pictures. Coding the automatic feature extractor for this feature proved close to impossible, with very limited reference material on the internet. Blue-white veil was, as shown, a feature not present in many pictures; however, it was possible to detect automatically.\\
\newline
Overall it will be difficult to use the trained classifier in the real world, as the random forest classifier resulted in 70\% of false negatives. For a life threatening disease such as melanoma, this outcome could be detrimental. The chosen three features (colour, asymmetry and blue-white veil) do therefore not seem feasible for this data set to safely classify melanoma.